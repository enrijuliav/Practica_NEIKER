\documentclass{article}
\usepackage{graphicx}


\begin{document}

\section{Personal opinion and difficulties}

Realizar estas prácticas ha sido bastante complicado, no por estas en si, sino por la programación del máster. Ha sido un asunto transversal, que ha atravesado el desarrollo todas las asignaturas, debido a la organización del tiempo. Al intentar seguir el programa y realizar todas las asignaturas a alvez, se evidencia una falta de horas en el día para avanzar todas las tareas. De esta manera, al ser unas prácticas online en la que debería trabajar a mi ritmo, la mayor parte del trabajo se ha ido posponiendo hasta que fuera inevitable.

Además de este asunto, también he tenido mi aprendizaje y dificultades particulares de esta praćtica. El principal asunto ha sido el síndrome de la página en blanco, podríamos llamarlo. En contré una gran dificultad a la hora de encontrar un paper sobre el que empezar a trabajar, al pensar que no iba a ser la elección idónea. Además de tener una serie de requisitos, que no se iban a cumplir perfectamente, había que ser realista, iba pasando de paper a paper buscando un tema en el que centrarme, lo cual era muy complicado ya que nada me convencía al 100\%.

A este problema se le sumó el hecho de que la metabolómica está bastante más atrasada que el resto de ómicas. Quizá sea por su naturaleza de estudio, ya que mientras que la proteómica y la genómica estudian secuencias de proteínas o ácidos nucléicos con estructuras bastante fijas y definidas, la metabolómica estudia la presencia y cantidad de diferentes metabolitos y sus interacciones, lo que es bastante más complicado de estudiar conceptual y realistamente, al haber una gran parte de confusión y ruido al pasar de secuencias de proteínas a metabolitos. Esto implica que hay pocas bases de datos de las que extraer información, hay pocos estudios que construyan unos sobre otros, superponiendo información y creando algo similar al genoma humano, por ejemplo. Por el momento solo tenemos bases de datos que contienen los resultados experimentales de cada grupo de investigación si eso.

Otro problema que surgió fue la mala replicabilidad de los papers. Una tendencia general de estos papers es de disponer de una mala replicabilidad, habiendo una falta, no solo del código empleado en sí, sino de versiones, parámetros y técnicas que osn esenciales para la obtención de los datos.

\section{Article:}

Metabolic coupling between soil aerobic methanotrophs and denitrifiers in rice paddy fields

\section{Abstract}

In rice paddy fields there is a lot of microbial denitrification, typicaly linked to oxidation of electron donors such as methane ($CH_4$) in \textit{anoxic} and \textit{hypoxic} conditions.

Whether and how \textit{aerobic} methane oxidation couples with denitrification in \textit{anoxic} paddy fields is unkown.

They are going to use different omic techniques to study that relation.

Results:
Postive relation between CH$_4$ oxidation and denitrification activities and genes.
CH$_4$ and methanotroph addition promote gene expresion in denitrification.
There is a high importance of intermidiates between aerobic CH$_4$ oxidation and denitrification

\section{Intro}

The justification of this paper is to understand the denitrification process to efficiently use nitrogen fertilizer and reduce greenhouse gases emissions (N$_2$O and CH$_4$).

It has been seen the relation between \textit{anoxic} and \textit{hypoxic} methane oxidation and denitrification. The problem is the taxa capable of those processes where not found in surface leyar of paddy fields (is partially oxic because of diffusion of oxygen from rice roots).\textbf{over 70\% of the produced CH$_4$ in hypoxic conditions is consumed by aerobic methanotrophs before escaping to the atmosphere.} It's been seen that CH$_4$ oxidation significantly promotes N removel via denitrification. However, the taxa and pathways in rice paddy fields remain unkown.

\begin{center}
\includegraphics[scale=0.3]{Methane-oxidation_Denitrification.png}
\end{center}

CH$_4$ may promote denitrification by cooperation between aerobic metanotrophs and denitrifiers via O$_2$ consumption and intermediates exchange.
Some metanotrophs were capable of partial denirification. \textbf{Hypothesis: Microbial aerobic CH4 oxidation may promote soil denitrification in rice paddies by a mutualism process.}

1.Field survey across china

2.Microcosm experiments (Coupling between CH$_4$ oxidation and denitrication genes: \textit{narG}, \textit{nirK}, \textit{nirS}, \textit{norB}, \textit{nosZi} and \textit{nosZII}; unther CH$_4$ and aerobic metanotrophs addition)

3. DNA-SIP with MAGs and $^{13}$C-metabolomics.

\section{Methods}

\subsection{Field survey}

\subsection{$^{13}$CH$_4$-DNA-stable isotope probing (SIP)}

DNA-SIP is a technique used to identify active microorganisms that assimilate particular carbon nutrients into cellular biomass.

You first start by incubating an environmental sambple with the stable isotope labelled compound, $^{13}$CH$_4$ in this case. Then you extract and purify the DNA, which lets you retrieve the labelled and unlabelled DNA for subsequent moelcular characterization.

\subsection{Microcosm experiments}

Microcosms are artificial, simplified ecosystems that are used to simulate and predict the behaviour of natural ecosystems under controlled conditions. Open or closed microcosms provide an experimental area for ecologists to study natural ecological processes.

\subsection{$^{13}$C-metagenome-assembled genoms}

\subsection{$^{13}$C-metabolomics}

Intermediates derived form $^{13}$CH$_4$.

\subsection{Stat analysis}

\section{Results}

\textbf{Relation between microbial CH4 oxidation and denitrification in 	field survey:} Wide variation in denitrification and methane oxidation activities and the abundance or its genes across China. A significant and positive relationship between denitrification rate and CH$_4$-oxidizing activity. Postitive correlation between \textit{pmoA} (methane monooxigenase) and \textit{nirK} and \textit{nirS}. The structural equation modeling (SEM) reconfirmed those results. Correlation network analyses indicated linkages between potential aerobil methane oxidation and denitrification taxa.

\textbf{Experimental coupling between aerobic CH4 oxidation and denitrification:}
To verify the influence of aerobic CH$_4$ oxidation on denitrification, we determined changes in N$_2$O emissions, NO$_3^-$ consumption and the expression of denitrification genes under CH4 addition in selected soils. The addition of methanotrophs also led to an overall increase in
N$_2$O emissions and NO$_3^-$-N consumption compared to the control.

\textbf{Microbial guilds associated with the coupling between aerobic CH4 oxidation and denitrification:} DNA-SIP experiments to identify key taxa associated with the coupling.

\textbf{Genes and metabolic pathways associated with the coupling between aerobic CH4 oxidation and denitrification:}these denitrifying MAGs also enriched genes responsible for the activation of carbonaceous organics such as methanol, formaldehyde, formate, short-chain fatty acids, as well as small molecular organic acids involved in pyruvate, TCA cycle.

\section{Replication}

\end{document}
