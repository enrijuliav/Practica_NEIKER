\documentclass{article}
\usepackage{graphicx}

\title{Paper NEIKER}
\author{Enrique Juliá}
\date{September 2025}

\begin{document}

\maketitle
\section{Abstract}

\section{Introduction}

In 2024 about 523.8 million metric tons of rice were consumed worldwide. Besides, over 50 percent of the world population depends on rice for about 80 percent of its food requirements. To meet this demand, a total of 200 million km$^2$ are destined to cultivating rice in paddy fields, more than 10\% of the cropland worldwide.
This huge buisness has it's downside, consuming large amounts of N-fertilizer and generating vast quantities of methane (CH$_4$), a greenhouse gas with 30 times more warming potential than CO$_2$. For those reasons, understanding the biological processes produced in these environments, is a key requirement for manteining the production of an essensial food while reducing its environmental cost.

Rice paddy fields are a very interesting ecosystem, containig, in the rainy months, a layer of water that creates an oxygenation gradient, perfect for the development of a huge variety of microorganisms.

\section{Methods}
\section{Results}
\section{Discussion}
\section{References}

\end{document}
